\section{Using the rospd-Package} \label{rospd}

The rospd-package (\textbf{R O}ptimal \textbf{S}plit \textbf{P}lot \textbf{D}esigns) is an implementation of the previously described algorithm. This chapter shows how to use the package function to create optimal semi-split-plot designs. The core part of the package are the two classes \textit{doeFactor} and \textit{doeDesign}. The \textit{doeDesign} class collects all information required to generate an optimal design. This includes a list of \textit{doeFactors} that should be investigated in the DoE. 

\subsection{The doeFactor class}

Objects of class \textit{doeFactor} represent factors in the DoE. \textit{doeFactor} is a S4 class with the following slots:

\begin{itemize}
\item \textbf{name:} The name of the factor as a character.
\item \textbf{type:} Either \textit{"continuous"} or \textit{"categorical"}. Continuous factors are numeric factors that could theoretically take any real number in a given range. Categorical factors use either character or numeric values. They are interpreted as nominal variables.
\item \textbf{levels:} For continuous factors the user specifies the range of the factor as a numeric vector containing minimum and maximum. For categorical factors a vector with all possible factor levels - either numeric or characters - is required.
\item \textbf{number.levels:} The number of levels is only required for continuous factors. This integer represents how many different levels are used during the design generation. E.g. for $number.levels = 3$ and $levels=c(100, 300)$ the factor levels that are used during the design updating are 100, 200 and 300. For $number.levels = 5$ and $levels=c(100, 300)$ the possible factor levels are 100, 150, 200, 250 and 300. The number of levels have a strong impact on the runtime of the algorithm.
\item \textbf{changes:} This defines if a factor is \textit{"easy"}-, \textit{"hard"}- or \textit{"semi.hard"}-to-change.
\item \textbf{semi.htc.group.size:} This is only required for SHTC-factors. This integer defines how many different factor levels can be used inside of one whole plot.
\end{itemize}

The factors of the current example can be declared the following way:

\begin{verbatim}
# pH - an ETC continuous factor
phFactor <- new("doeFactor",
                name="pH",
                type="continuous",
                levels = c(3,12),
                number.levels = as.integer(2),
                changes="easy"
                )

# time - an ETC continuous factor
timeFactor <- new("doeFactor",
                name="time",
                type="continuous",
                levels = c(10,20),
                number.levels = as.integer(2),
                changes="easy"
                )

# solvent - a SHTC categorical factor
solventFactor <- new("doeFactor",
                     name="solvent",
                     type="categorical",
                     levels=c("A", "B", "C",
                     		  "D", "E", "F"),
                     changes="semi.hard",
                     semi.htc.group.size=
                      as.integer(4)
                     )
\end{verbatim}

\subsection{ The doeDesign class}

The \textit{doeDesign}-class is the heart of rospd. It stores all information required to generate an optimal design and will contain the final design table afterwards as well. This way any \textit{doeDesign}-object can serve as a documentation of how the design was created. \textit{doeDesign} is a S4-class. As it uses a lot of different slots only the required ones will be discussed here.

\begin{itemize}
\item \textbf{factors:} A list of all factors that are used in the DoE as objects of class \textit{doeFactor}. This list can contain one or more items.
\item \textbf{whole.plot.structure:} The whole plot structure is a data.frame representing when changes can be made for all HTC- and SHTC-factor. This data.frame needs to contain one column for each HTC- and SHTC-factor using the factor names as column names. The whole plots of each factor are represented by index numbers starting at 1 for each factor.
\item \textbf{number.runs:} The number of experiments as \textit{integer}.
\item \textbf{design.model:} The design model as a \textit{formula}. The formula is supposed to contain only the fixed effects part of the model as the random effects part is already defined by the \textbf{whole.plot.structure}.
\item \textbf{optimality.function:} a \textit{function} that can be used to calculate the optimality of a given design. The algorithm will use the \textit{doeDesign}-object as the only argument of the function. Making use of the slots \textbf{design.matrix}, \textbf{design.model} and \textbf{variance.ratio} of the class \textbf{doeDesign} should allow to calculate most optimality criteria. The rospd package provides a predefined function to calculate D-Optimality. The function is a wrapper of the C-implementation of the skpr-package \cite{skprpkg}.
\item \textbf{variance.ratio:} The expected ratio of $\sigma^2$ (residual variance) and $\sigma^2_w$ (between whole plot variance). This is required for the calculation of some optimality criteria. By default a variance ratio of 1 is used.
\end{itemize}

There is a function \textbf{GenerateNewDoEDesign} that should be used to initialize a new \textit{doeDesign}-object. A \textit{doeDesign}-object for the current example might look like this:

\begin{verbatim}
  doeSpecification <- GenerateNewDoeDesign(
      factors = list(
      				solventFactor, 
      				phFactor, 
      				timeFactor),
      whole.plot.structure = 
      		data.frame(solvent=
      			rep(1:6, each=10)),
      number.runs = as.integer(60),
      design.model = ~solvent+pH+time,
      optimality.function = DOptimality
  )
\end{verbatim}

\subsection{Generating Optimal Designs}

The function \textbf{GenerateOptimalDesign} generates an optimal design for the given specification. The function uses the following arguments:

\begin{itemize}
\item \textbf{doeSpec:} The doe specification as an object of class \textit{doeDesign}.
\item \textbf{random.start:} This argument controls how many different random starting designs are used. The default setting is 1. The function will return a list of one doeDesign for each random start. The designs in this list will be sorted from best to worst in terms of their optimality.
\item \textbf{max.iter:} The maximum number of iterations. One iterations updates all cells of the design table once. 
\end{itemize}

The following code generate a D-optimal semi split-plot design.

\begin{verbatim}
  optimalDesign <- 
    GenerateOptimalDesign(doeSpecification)
\end{verbatim}

Table \ref{dosspd} shows a subset of the generated design.

\begin{table*}[!h]
\centering
\begin{tabular}{rlrr|rlrr}
  \hline
Whole Plot & Solvent & pH & Time & Whole Plot & Solvent & pH & Time \\ 
  \hline
  \hline
1 & B & 12 & 20 &   5 & D & 12 & 20 \\ 
1 & B & 3 & 10 &   5 & D & 3 & 10 \\ 
1 & F & 3 & 20 &   5 & E & 3 & 10 \\ 
 1 & C & 12 & 10 &   5 & E & 12 & 10 \\ 
 1 & C & 3 & 20 &   5 & E & 12 & 20 \\ 
 1 & C & 12 & 20 &   6 & A & 12 & 20 \\ 
 1 & D & 3 & 10 &   6 & A & 3 & 10 \\ 
 1 & F & 12 & 10 &   6 & A & 3 & 20 \\ 
 1 & D & 12 & 10 &   6 & B & 3 & 20 \\ 
 1 & F & 3 & 20 &   6 & B & 3 & 10 \\ 
 2 & E & 12 & 20 &   6 & B & 12 & 10 \\ 
 2 & A & 12 & 20 &   6 & F & 12 & 20 \\ 
 2 & A & 12 & 10 &   6 & E & 3 & 20 \\ 
 2 & E & 3 & 20 &   6 & E & 12 & 10 \\ 
 2 & B & 3 & 20 &   6 & F & 12 & 10 \\ 
   \hline
\end{tabular}
\caption{Part of the Semi Split Plot Design} \label{dosspd}
\end{table*}